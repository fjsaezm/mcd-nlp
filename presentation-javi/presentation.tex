\documentclass[aspectratio=169]{beamer}



% OPCIONES DE BEAMER

\definecolor{Maroon}{cmyk}{0, 0.87, 0.88, 0.1}
\definecolor{teal}{rgb}{0.0, 0.45, 0.45}

\usetheme[block=fill,numbering=fraction,, subsectionpage=progressbar, titleformat section=smallcaps]{metropolis}
\setbeamertemplate{frametitle continuation}[roman]
\setbeamertemplate{section in toc}[balls numbered]
\setbeamertemplate{subsection in toc}[subsections unnumbered]
%\setsansfont[BviejoFont={Fira Sans SemiBold}]{Fira Sans Book}  % Increase font weigth
\widowpenalties 1 10000
\raggedbottom

% COLORES
\setbeamercolor{palette primary}{bg=teal}
\setbeamercolor{progress bar}{use=Maroon, fg=Maroon}

% PAQUETES
\usepackage{bm}
\usepackage{xcolor}
\colorlet{shadecolor}{blue!15}
\usepackage{framed}
\usepackage{amsthm}
\usepackage[utf8]{inputenc}
\usepackage[spanish, es-noshorthands]{babel}
\usepackage{subfig}
\usepackage{graphicx}
\usepackage{minted}
\usepackage{ upgreek }



% Macros
\newcommand{\bx}{\bm{x}}
\newcommand{\bX}{\bm{X}}
\newcommand{\bw}{\bm{w}}
\newcommand{\bW}{\bm{W}}
\newcommand{\bz}{\bm{z}}
\newcommand{\bZ}{\bm{Z}}
\newcommand{\bv}{\bm{v}}
\newcommand{\bV}{\bm{V}}
\newcommand{\bH}{\bm{H}}
\newcommand{\bh}{\bm{h}}
\newcommand{\bSigma}{\bm{\Sigma}}
\newcommand{\bpi}{\bm{\pi}}
\newcommand{\bLambda}{\bm{\Lambda}}
\newcommand{\bmu}{\bm{\mu}}
\newcommand{\btheta}{\bm{\theta}}
\newcommand{\bnu}{\bm{\nu}}
\DeclareMathOperator*{\argmax}{arg\,max}
\DeclareMathOperator*{\argmin}{arg\,min}
\newcommand\E[2]{\mathbb{E}_{#1}\left[#2\right]}
\newcommand\KL[2]{D_{KL}\Big(#1 \bigm|\bigm| #2\Big)}
\newcommand{\bigCI}{\mathrel{\text{\scalebox{1.07}{$\perp\mkern-10mu\perp$}}}}
\newcommand{\bigCD}{\centernot{\bigCI}}
\newcommand{\X}{\mathcal{X}}
\newcommand{\R}{\mathbb{R}}
\usepackage{pgfplots}

\newcommand{\norm}[1]{\left\lVert#1\right\rVert}
\newcommand{\abs}[1]{\left\lvert#1\right\rvert}
\newcommand{\ps}{x^+}
\newcommand{\ns}{x^-}


% TikZ
\usepackage{tikz}

\usepackage{arydshln}

\captionsetup[subfloat]{labelformat=empty}

\newtheorem{defi}{Definición}
\newtheorem{prop}{Proposición}
\newtheorem{nth}{Teorema}
\newtheorem{cor}{Corolario}




\usetikzlibrary{arrows.meta,
chains,
positioning}

\newcommand\Fontvi{\fontsize{8}{7.2}\selectfont}

\title{Adversarial Training with Contrastive Learning in NLP}
\subtitle{Máster en Ciencia de Datos}
\date{\today}
\author{Francisco Javier Sáez Maldonado}
\institute{Procesamiento de Lenguaje Natural \\\\\\ \emph{Escuela Politécnica Superior} \\ \emph{Universidad Autónoma de Madrid}}

\usepackage[absolute,overlay]{textpos}


\graphicspath{{../thesis/media/}}


\begin{document}
  \maketitle


  \begin{frame}{Índice}
    \begin{columns}
      \begin{column}{0.5\textwidth}
         % Inferencia estadística\\
         % \quad Enfoques\\
         \textbf{1. Teoría de la información}\\
         %\quad Entropía\\
         \quad Información mutua\\
         \quad Cotas inferiores\\
         \vspace*{0.2cm}
         \textbf{2. Aprendizaje contrastivo}\\
         \quad Estimación del ruido contrastiva\\
         \quad Contrastive predictive coding\\
         \quad Pérdida usando tripletas\\
       \end{column}
       \begin{column}{0.5\textwidth}
         \textbf{3. Nuevos marcos de trabajo}\\
         \quad SimCLR \\
         \quad Bootstrap your own latent\\
         \vspace*{0.2cm}
         \textbf{4. Experimentación}\\
         \quad Objetivos\\
         \quad Experimentos con SimCLR\\
         \quad Experimentos con BYOL\\
       \end{column}
     \end{columns}
  \end{frame}

  

  \begin{frame}{Motivación}

  \centering
  \begin{columns}
      \begin{column}{0.5\textwidth}
        \begin{center}
        \textbf{Dato}
        \end{center}
        \[
      \left(0.1,0,2,1,0,0.5,2.4,5 \right)  
     \]
      \end{column}
      \begin{column}{0.5\textwidth}  %%<--- here
        \begin{center}
          \textbf{Etiqueta}
          \end{center}
        \[
        \text{Perro}  
        \]

      \end{column}
    \end{columns}


    \pause 
    
    \begin{shaded}
      Sea $x \in \R^d$ un vector de entrada a un modelo de aprendizaje automático. Una \emph{representación} $\tilde{x} \in \R^n$ es otro vector de menor dimensión que comparte información o características con $x$.
    \end{shaded}


    {\color{Maroon}\textbf{Objetivo}:} extraer \textbf{representaciones} que sean buenas en general para \textbf{tareas posteriores}.
  \end{frame}

  
\section{Introducción}


  \section{Aprendizaje contrastivo}


      

  
  \begin{frame}{Pérdida contrastiva y cota inferior contrastiva}
    
    \begin{defi}[Pérdida contrastiva]
      Sea \(X =\{x^*,x_1,\dots,x_{N-1}\}\) un conjunto de \(N\) ejemplos donde $x^*$ ha sido extraido de la distribución conjunta \(P(x,z)\) y le resto han sido extraídos del producto de las distribuciones marginales \(P(x),P(z)\). Se define entonces la función de pérdida contrastiva como  
      \[ 
        \ell(\theta) = - E_X \left[ \log \frac{h_\theta(x^*,z)}{\sum_{x \in X}h_\theta(x,z)}\right]. 
        \]
    \end{defi}

  
  \end{frame}




  
  
  \begin{frame}{Tercer experimento y resultados finales}

  \end{frame}

  
  
  \begin{frame}{Experimento con BYOL}

  
  
  
  \end{frame}
  

  


  
  \begin{frame}{Conclusiones}

    \begin{itemize}
      \item El uso de la teoría de la información proporciona un buen punto de partida para el aprendizaje de representaciones.
      \item El aprendizaje contrastivo ha probado ser la mejor forma de obtener representaciones que son útiles en tareas posteriores.
      \item Ambos marcos de trabajo probados obtienen buenos resultados en la adaptación a conjuntos de datos más pequeños.
    \end{itemize}
  \end{frame}
  
  \appendix

  \begin{frame}[noframenumbering,standout]
    Gracias por su atención
  \end{frame}




\end{document}
