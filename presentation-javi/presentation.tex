\documentclass[aspectratio=169]{beamer}



% OPCIONES DE BEAMER

\definecolor{Maroon}{cmyk}{0, 0.87, 0.88, 0.1}
\definecolor{teal}{rgb}{0.0, 0.45, 0.45}

\usetheme[block=fill,numbering=fraction,, subsectionpage=progressbar, titleformat section=smallcaps]{metropolis}
\setbeamertemplate{blocks}[rounded][shadow=false]
\setbeamertemplate{frametitle continuation}[roman]
\setbeamertemplate{section in toc}[balls numbered]
\setbeamertemplate{subsection in toc}[subsections unnumbered]
%\setsansfont[BviejoFont={Fira Sans SemiBold}]{Fira Sans Book}  % Increase font weigth
\widowpenalties 1 10000
\raggedbottom

% COLORES
\setbeamercolor{palette primary}{bg=teal}
\setbeamercolor{progress bar}{use=Maroon, fg=Maroon}

% PAQUETES
\usepackage{bm}
\usepackage{xcolor}
\colorlet{shadecolor}{blue!15}
\usepackage{framed}
\usepackage{amsthm}
\usepackage[utf8]{inputenc}
\usepackage[spanish, es-noshorthands]{babel}
\usepackage{subfig}
\usepackage{graphicx}
\usepackage{minted}
\usepackage{ upgreek }



% Macros
\newcommand{\bx}{\bm{x}}
\newcommand{\bX}{\bm{X}}
\newcommand{\bw}{\bm{w}}
\newcommand{\bW}{\bm{W}}
\newcommand{\bz}{\bm{z}}
\newcommand{\bZ}{\bm{Z}}
\newcommand{\bv}{\bm{v}}
\newcommand{\bV}{\bm{V}}
\newcommand{\bH}{\bm{H}}
\newcommand{\bh}{\bm{h}}
\newcommand{\bSigma}{\bm{\Sigma}}
\newcommand{\bpi}{\bm{\pi}}
\newcommand{\bLambda}{\bm{\Lambda}}
\newcommand{\bmu}{\bm{\mu}}
\newcommand{\btheta}{\bm{\theta}}
\newcommand{\bnu}{\bm{\nu}}
\DeclareMathOperator*{\argmax}{arg\,max}
\DeclareMathOperator*{\argmin}{arg\,min}
\newcommand\E[2]{\mathbb{E}_{#1}\left[#2\right]}
\newcommand\KL[2]{D_{KL}\Big(#1 \bigm|\bigm| #2\Big)}
\newcommand{\bigCI}{\mathrel{\text{\scalebox{1.07}{$\perp\mkern-10mu\perp$}}}}
\newcommand{\bigCD}{\centernot{\bigCI}}
\newcommand{\X}{\mathcal{X}}
\newcommand{\R}{\mathbb{R}}
\usepackage{pgfplots}

\newcommand{\norm}[1]{\left\lVert#1\right\rVert}
\newcommand{\abs}[1]{\left\lvert#1\right\rvert}
\newcommand{\ps}{x^+}
\newcommand{\ns}{x^-}


% TikZ
\usepackage{tikz}

\usepackage{arydshln}

\captionsetup[subfloat]{labelformat=empty}

\newtheorem{defi}{Definición}
\newtheorem{prop}{Proposición}
\newtheorem{nth}{Teorema}
\newtheorem{cor}{Corolario}
\newtheorem{ex}{Ejemplo}

\definecolor{studentbrown}{RGB}{124,71,50}
\AtBeginEnvironment{ex}{
    \setbeamercolor{block title}{use=example text,fg=white,bg=example text.fg!75!black}
    \setbeamercolor{block body}{parent=normal text,use=block title example,bg=block title example.bg!10!bg}
}

\usetikzlibrary{arrows.meta,
chains,
positioning}

\newcommand\Fontvi{\fontsize{8}{7.2}\selectfont}

\title{Adversarial Training with Contrastive Learning in NLP}
\subtitle{Procesamiento de Lenguaje Natural}
\date{\today}
\author{Francisco Javier Sáez Maldonado}
\institute{Máster en Ciencia de Datos \\\\\\ \emph{Escuela Politécnica Superior} \\ \emph{Universidad Autónoma de Madrid}}

\usepackage[absolute,overlay]{textpos}


\begin{document}
  \maketitle


  \begin{frame}{Índice}
    \tableofcontents
  \end{frame}

  

  \begin{frame}{Introducción}

    \begin{itemize}
      \item \textbf{Tarea}: Modelado del lenguage (LM) y Traducción automática (NMT)
      \pause 
      \item \textbf{Objetivo}: Conseguir modelos que sean más robustos semánticamente:
      \[
      \text{Inputs parecidos} \implies \text{ Outputs parecidos}  
      \]
    \end{itemize}
  \end{frame}
  

  \section{Herramientas}

  \subsection{Adversarial Training}
  \begin{frame}{Adversarial Training}

    \begin{defi}[Adversarial Learning]
      Técnica usada en el aprendizaje automático para, usando información sobre un modelo, crear ataques maliciosos para causar fallos en el modelo
    \end{defi}
    \pause

    \begin{defi}[Adversarial example]
      Ejemplo diseñado para engañar al modelo, creado introduciendo una \emph{perturbación} en un ejemplo original.
    \end{defi}

    \begin{center}
    ¿ Cómo ayuda el aprendizaje adversario a nuestros modelos ?
    \end{center}

  \end{frame}

  \begin{frame}{Ejemplos adversarios en NLP}
    Ejemplos de técnicas:

    \begin{itemize}
      \item Visuales
      \item omega
    \end{itemize}
  \end{frame}

  \begin{frame}{Ejemplos adversarios}
    Dada una secuencia \(s = \{x_1,\dots,x_T\}\) de tokens
    \begin{enumerate}
      \item Creamos una representación embebida en un espacio continuo 
      \[
        \mathbf{E}x_i = e_i.  
      \]
      \item Añadimos una pequeña perturbación en el embedding
      \[
      e_i' = e_i - \epsilon \frac{g}{\norm{g}_2},   
      \]
      siendo \(g = \nabla_{e_i}J(s,\theta)\) y \(J\) la función de coste.
    \end{enumerate}

    Función de coste actual:
    \[
    \mathcal J(\theta) = \sum_s \mathcal L(s,\theta) + \alpha \sum_{s'} \mathcal L_{adv} (s',\theta), \quad \alpha \in [0,1].  
    \]
  \end{frame}

  \subsection{Contrastive Learning}

  \begin{frame}{Contrastive Learning}
    
      {\color{Maroon}\textbf{Idea}:} Acercar las representaciones de ejemplos positivos (de la misma clase) y alejar las de los ejemplos negativos (resto de ejemplos).
    

    \begin{ex}
      Original: Elefante. Positivo: Hipopótamo. Negativo: Pistola.
    \end{ex}

    \begin{defi}[Pérdida contrastiva]
      Sean \(a_i\) las entradas originales, \(p_{a_i}\) ejemplos positivos y \(n_{a}\) ejemplos negativos. Se define la pérdida contrastiva como:
      \[
      \mathcal L_{cont} = - \sum_{a_i \in A} \log \frac{\exp(a_i \cdot p_{a_i}/\tau)}{\sum_{n_a \in A - \{a_i\} \exp(a_i \cdot n_a / \tau)}}  
      \]
    \end{defi}
  \end{frame}


  \section{Framework}




  
  \begin{frame}
  \end{frame}

  \section{Resultados}




  
  \begin{frame}{Conclusiones}

    \begin{itemize}
      \item El uso de la teoría de la información proporciona un buen punto de partida para el aprendizaje de representaciones.
      \item El aprendizaje contrastivo ha probado ser la mejor forma de obtener representaciones que son útiles en tareas posteriores.
      \item Ambos marcos de trabajo probados obtienen buenos resultados en la adaptación a conjuntos de datos más pequeños.
    \end{itemize}
  \end{frame}
  
  \appendix

  \begin{frame}[noframenumbering,standout]
    Gracias por su atención
  \end{frame}




\end{document}
